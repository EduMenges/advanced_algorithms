\documentclass[11pt]{article}
\usepackage[
    inner=3cm,
    outer=2cm,
    top=3cm,
    bottom=2cm
]{geometry}
\usepackage[brazil]{babel}
\usepackage{graphicx}
\usepackage{fontspec}
\usepackage{float}
\usepackage{listings}
\setlength{\parskip}{10pt}
\setmainfont{texgyreheros}[
    UprightFont = *-regular,
    BoldFont = *-bold,
    ItalicFont = *-italic,
    BoldItalicFont = *-bolditalic,
    Extension = .otf
]
\lstset{
	language=C++,
	keepspaces=true
}

\title{Laboratório 4 — Emparelhamentos}
\author{Eduardo Menges Mattje}

\begin{document}
\maketitle

\section{Introdução}
O algoritmo húngaro resolve problemas de emparelhamento para grafos bi-partidos ao buscar repetidamente por caminhos aumentantes de forma a tornar valores potenciais iguais à atribuição que minimiza (ou maximiza) a soma dos emparelhamentos.
A forma de encontrar os caminhos aumentantes escolhida para esta implementação foi a de Johnson, que utiliza Dijkstra para encontrar os menores caminhos.

\section{Detalhes de implementação}
\label{sec:implementation-details}

O algoritmo é iterativo, atribuindo um trabalho a um trabalhador a cada iteração, chegando, portanto, na solução somente na última iteração.

Há um trabalhador extra, que se liga a todos os trabalhos no começo das iterações e inicializa a pilha (implementada com \lstinline|std::priority_queue|) do Dijkstra.
Então, para cada outro trabalhador, empilha com as distâncias atualizadas considerando seus potenciais (valores do dual) e a relaxação em relação ao trabalhador atual, somente se essa for menor do que a distância já existente.

O Dijkstra se encerra ao encontrar o primeiro trabalhador que não possui trabalho associado.
Então, os potenciais são atualizados e o custo é somado com este.

A representação de grafos escolhida foi a de matriz de adjacência devido aos grafos de entrada serem presumidamente completos.

\section{Ambiente de teste}

A máquina de teste possui Windows 11, um processador AMD Ryzen 5 7600X e 32 GB de memória RAM. O compilador utilizado foi o Clang, versão 19.1.5, com todas as otimizações padrões habilitadas e utilizando a especificação do C++ 23.

\section{Plano de teste}

O gerador utilizado para gerar os grafos toma uma entrada $n$ e gera uma matriz de tamanho $n \times n$ com valores uniformemente distribuídos no intervalo $[0, n \cdot n]$.

Foram gerados $10$ testes para cada $n$ pertencente ao intervalo $[1000, 20000]$, incrementando $1000$ a cada passo. Para cada teste, mensurou-se:

\begin{itemize}

\item o tempo de execução;
\item o número de iterações do laço principal;
\item a quantidade de operações do heap;
\item a quantidade de relaxações realizadas.

\end{itemize}

\section{Resultados}

\subsection{Tempo de execução}

\begin{figure}[H]
    \centering
    \caption{Tempo de execução em função de $n$}
    \label{fig:n_time}
    \includegraphics[width=0.8\linewidth]{graphs/n_time}
\end{figure}

Utilizando o algoritmo de Johnson para caminhos aumentantes, o algoritmo húngaro tem como complexidade $O(n(m + n \log n))$.
Como o gerador gera matrizes quadradas, $m = n^2$, logo a complexidade esperada é $O(n^3 + n^2 \log n)$.
Uma regressão para a equação $an^3 + bn^2 \log n$ cabe aos dados, conforme apontado na figura \ref{fig:n_time}, tendo o coeficiente $R^2 = 0,9987$ muito próximo de 1.

\subsection{Laço principal}

\begin{figure}[H]
	\centering
	\caption{Iterações no laço principal em relação a $n$}
	\label{fig:main_loop}
	\includegraphics[width=0.8\linewidth]{graphs/main_loop}
\end{figure}

Devido à implementação iterativa do algoritmo, o crescimento das iterações no laço principal ocorre de maneira linear a $n$, conforme esperado.

\subsection{Operações no heap}
\label{sec:heap_ops}

\begin{figure}[H]
	\centering
	\caption{Operações no heap em função de $n$}
	\label{fig:heap_ops}
	\includegraphics[width=0.8\linewidth]{graphs/heap_ops}
\end{figure}

Conforme discutido em \ref{sec:implementation-details}, o Dijkstra para ao encontrar o primeiro trabalhador livre, e na $k$-ésima iteração, há $n - k + 1$ trabalhadores livres.
Desta forma, o número esperado de operações \lstinline|pop()| até encontrarmos o primeiro trabalhador livre é $\frac{n}{n - k + 1}$, o que, nas primeiras 50\% das iterações é $<= 2$, e mesmo nos 10\% finais o valor é $<= 10$, o que seria um tempo constante $\Theta(1)$.

Como percorremos todos os outros $n$ vizinhos no processo de relaxação, e como cada relaxação pode gerar um \lstinline|push()| no heap, temos $O(n)$ operações no heap.
Por fim, o laço principal do algoritmo faz $O(n^2)$ operações no heap, conforme observado nos dados coletados.

\subsection{Relaxações}

\begin{figure}[H]
\centering
\caption{Relaxações bem-sucedidas em função de $n$}
\label{fig:relaxations}
\includegraphics[width=0.8\linewidth]{graphs/relaxations}
\end{figure}

As relaxações são contadas somente quando bem-sucedidas ou seja, se realmente melhoram a distância, e só ocorrem caso o primeiro \lstinline|pop()| não ocorra com um trabalhador livre. Seguindo um raciocínio parecido com o item \ref{sec:heap_ops}, sabemos que a probabilidade de isso ocorrer é baixa, e cada vez que ocorre são realizadas no máximo $n$ relaxações.
Deste modo, $O(1) * O(n) = O(n)$, e contando as iterações do laço principal temos $O(n^2)$.

\section{Conclusão}

Com essa mensuração e os testes com matrizes quadradas, conseguimos observar que a complexidade teórica é respeitada para o caso médio, o que é reforçado pelos testes de regressão realizados e a probabilidade de emparelhamento dos trabalhadores livres.

\end{document}

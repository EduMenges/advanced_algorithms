\documentclass[11pt]{article}
\usepackage[
    inner=3cm,
    outer=2cm,
    top=3cm,
    bottom=2cm
]{geometry}
\usepackage[brazil]{babel}
\usepackage{graphicx}
\usepackage{fontspec}
\usepackage{float}
%\usepackage{listings}
\setlength{\parskip}{10pt}
\setmainfont{texgyreheros}[
    UprightFont = *-regular,
    BoldFont = *-bold,
    ItalicFont = *-italic,
    BoldItalicFont = *-bolditalic,
    Extension = .otf
]

\title{Laboratório 4 — Emparelhamentos}
\author{Eduardo Menges Mattje}

\begin{document}
\maketitle

\section{Introdução}

O algoritmo húngaro resolve problemas de emparelhamento para grafos bi-partidos ao buscar repetidamente por caminhos aumentantes de forma a tornar valores potenciais iguais à atribuição que minimiza (ou maximiza) a soma dos emparelhamentos.
A forma de encontrar os caminhos aumentantes escolhida para esta implementação foi a de Johnson, que utiliza Dijkstra para encontrar os menores caminhos.

\section{Detalhes de implementação}

O algoritmo é iterativo, atribuindo um trabalho a um trabalhador a cada iteração.

Há um trabalhador extra, que se liga a todos os trabalhos no começo das iterações e inicializa a pilha (implementada com std::priority\_queue) do Dijkstra.
Então, para cada outro trabalhador, empilha com as distâncias atualizadas considerando seus potenciais (valores do dual) e a relaxação em relação ao trabalhador atual, somente se essa for menor do que a distância já existente.

O Dijkstra se encerra ao encontrar o primeiro trabalhador que não possui trabalho associado.
Então, os potenciais são atualizados e o custo é somado com este.

A representação de grafos escolhida foi a de matriz de adjacência devido aos grafos de entrada serem presumidamente completos.

\section{Ambiente de teste}

A máquina de teste possui Linux Mint 22.1, processador Intel i5-12450H e 16GB de memória RAM.
O compilador utilizado foi o GCC, versão 14.2, com todas as otimizações padrões habilitadas, e a implementação da biblioteca padrão do C++ é a libstdc++ versão 14.

\section{Plano de teste}

O gerador utilizado para gerar os grafos toma uma entrada $n$ e gera uma matriz $N \times N$ com valores no intervalo $[0, n \cdot n]$.

Foram gerados $10$ testes para cada $n$ pertencente ao intervalo $[1000, 20000]$, incrementando $1000$ a cada passo. Para cada teste, mensurou-se o tempo de execução.

\section{Resultados}

\begin{figure}[H]
    \centering
    \caption{Tempo de execução em função de $n$}
    \label{fig:n_time}
    \includegraphics[width=0.8\linewidth]{graphs/n_time}
\end{figure}

Utilizando o algoritmo de Johnson para caminhos aumentantes, o algoritmo húngaro tem como complexidade $O(n(m+n \log n))$.
Como o gerador gera matrizes quadradas, $n = m$, logo a complexidade é $O(n^2 + n^2 \log n)$.
Uma regressão não linear com coeficientes $a$ e $b$ para esses termos cabe aos dados, conforme apontado na figura \ref{fig:n_time}.

\section{Conclusão}

Com essa mensuração, conseguimos observar que a complexidade teórica é respeitada.

\end{document}

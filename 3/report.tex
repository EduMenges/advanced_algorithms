\documentclass[11pt]{article}
\usepackage[
    inner=3cm,
    outer=2cm,
    top=3cm,
    bottom=2cm
]{geometry}
\usepackage[brazil]{babel}
\usepackage{graphicx}
\usepackage{fontspec}
\usepackage{float}
\setlength{\parskip}{10pt}
\setmainfont{texgyreheros}[
    UprightFont = *-regular,
    BoldFont = *-bold,
    ItalicFont = *-italic,
    BoldItalicFont = *-bolditalic,
    Extension = .otf
]

\title{Laboratório 3 — Problema dos torneios}
\author{Eduardo Menges Mattje}

\begin{document}
\maketitle

\section{Introdução}

O problema dos torneios em determinar se a primeira equipe poderá ganhar um torneio, sabendo os jogos seguintes entre todas as equipes e a quantidade de jogos já ganhos.

Esse problema pode ser reduzido para um fluxo máximo.

Dado $w_1$ o número de jogos já ganhos pela equipe $1$, $r_1$ o número de partidas restantes da equipe $1$, sabemos que a equipe $1$ vencerá no máximo $w_1 + r_1$ jogos.
Dessa forma, precisamos que o resto das equipes $i \in [2, n]$ ganhe no máximo $m_i = (w_1 + r_1) - w_i - 1$ vezes (excluindo os empates).

Para a montagem do grafo $G$, teremos um nó fonte $s$ e um nó sumidouro $t$.
Cada jogo também ganhará um nó, conectado à $s$ com peso $g(i,j)$, e às respectivas equipes $i$ e $j$ com capacidade infinita.
As equipes, por sua vez, se conectam ao sumidouro $t$ com capacidade $w_1$.
E equipe $1$ pode vencer se há um fluxo máximo saturado.

\section{Plano de teste}

O gerador de testes disponibilizado pelo professor permite a manipulação:

\begin{itemize}
	\item da quantidade de equipes;
    \item da quantidade máxima de jogos entre as equipes;
    \item da distribuição de jogos que já foram jogados;
    \item do viés de vencimento da equipe 1.
\end{itemize}

O que será avaliado será:

\begin{enumerate}
	\item o tempo de execução em função da quantidade de equipes;
	\item a probabilidade de a equipe $1$ vencer em função do viés e dos jogos já vencidos.
\end{enumerate}

Assim, os casos foram para 1 gerados no intervalo $[500, 5000]$, incrementando $500$ equipes a cada iteração, mantendo os demais parâmetros com os padrões do gerador.

Já os casos para 2 foram gerados fixando a quantidade de equipes em $250$, a quantidade máxima de jogos em $3$ e os jogos já jogados variando de $0.0$ a $1.0$, com incrementos de $0.1$, o viés da equipe $1$ variando de $-1.0$ a $1.0$ incrementando $0.1$ por iteração. $15$ casos de teste foram gerados por configuração.

\section{Detalhes de implementação}

A implementação do max flow utilizada foi a proposta por Dinic, em vista de que testes experimentais mostraram demasiada demora no tempo execução com as implementações Ford-Fulkerson e Edmonds-Karp. A implementação de Dinic possui complexidade $O(V^2E)$.

Devido à grande quantidade de nós e esparsidade de arestas, a representação de grafos escolhida foi a de lista de adjacência.

\section{Ambiente de teste}

A máquina de teste possui Windows 11, um processador AMD Ryzen 5 7600X e 32 GB de memória RAM. O compilador utilizado foi o Clang, versão 19.1.1, com todas as otimizações padrões habilitadas.

\section{Resultados}

\subsection{Tempo}

Podemos observar que o tempo de execução toma uma forma polinomial em relação à quantidade de equipes.

\begin{figure}[H]
	\centering
	\caption{Tempo de execução em função da quantidade de equipes}
	\label{fig:teams}
	\includegraphics[width=0.8\linewidth]{graphs/teams}
\end{figure}

Isso é condizente com a complexidade da implementação de Dinic, pois sua complexidade $O(V^2E)$ implica aumento quadrático no tempo conforme a quantidade de vértices cresce, e como há um vértice para cada possível jogo entre as equipes e essa quantidade cresce de forma quadrática ($vertices = \frac{(n - 1) \cdot (n - 2)}{2}$), o tempo aumenta numa relação polinomial de grau 4.

\subsection{Bias e taxa de vencimento}

Em vista de testarmos 2 variáveis, escolheu-se representar os dados com um \textit{heatmap}.

\begin{figure}[H]
    \centering
    \caption{Probabilidade de vencimento em função do viés e jogos já vencidos}
    \label{fig:bias_wins}
    \includegraphics[width=0.8\linewidth]{graphs/heatmap}
\end{figure}

Podemos observar que, mesmo em cenários sem viés ou com viés contra a equipe 1, ela ainda é capaz de ganhar, desde que haja uma quantidade suficiente de jogos ainda não jogados.

\section{Conclusão}

Através da redução proposta, foi possível resolver o problema dos torneios, com uma complexidade de acordo com o algoritmo de Dinic usado como a implementação do fluxo máximo.

\end{document}